\RequirePackage{fix-cm}

%\documentclass{svjour3}                     % onecolumn (standard format)
%\documentclass[smallcondensed]{svjour3}     % onecolumn (ditto)
\documentclass[smallextended]{svjour3}       % onecolumn (second format)
%\documentclass[twocolumn]{svjour3}          % twocolumn

\usepackage{amsmath}
\usepackage{graphicx}
\usepackage{tabto}
%\usepackage{latexsym}
%\usepackage{mathptmx}      % use Times fonts if available on your TeX system

\smartqed  % flush right qed marks, e.g. at end of proof

% Math definitions
\newcommand{\argmin}{\operatornamewithlimits{argmin}}
\newcommand{\tightoverset}[2]{\mathop{#2}\limits^{\vbox to -.6ex{\kern-0.75ex\hbox{$#1$}\vss}}}
%\newcommand{\tab}[1]{%
%    \settowidth{\tabcont}{#1}
%    {\makebox[0.15\linewidth][l]{#1}\ignorespaces}
%}%

% Insert the name of "your journal" with
\journalname{Neuro-Inspired Computational Elements Workshop}

\begin{document}

\title{
%\thanks{Grants or other notes
%about the article that should go on the front page should be
%placed here. General acknowledgments should be placed at the end of the article.}
}

%\subtitle{}

%\titlerunning{Short form of title}        % if too long for running head

\author{}


\authorrunning{}% if too long for running head


\institute{}
}

\date{}

\begin{displaymath}
    T_{\lambda}(u_{m}(t)) = \left\{
    \begin{aligned}
        0,\;\; &u_{m}(t)\; \leq\; \lambda \\
        u_{m}(t),\;\; &u_{m}(t)\; >\; \lambda
    \end{aligned}
    \right.
\label{thresholdfunc}
\end{displaymath}

\end{document}

